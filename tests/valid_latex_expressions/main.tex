\documentclass{article}

\setlength{\topmargin}{-.5in}
\setlength{\textheight}{9in}
\setlength{\oddsidemargin}{0in}
\setlength{\textwidth}{6.5in}

%\usepackage{graphicx} % Required for inserting images
\usepackage{amsmath}

\usepackage{hyperref}


\title{Demonstration that expressions are valid \LaTeX}
\author{ben.is.located@gmail.com}
\date{\today}

\begin{document}

\maketitle

\section{Introduction}
This document validates that expressions are \LaTeX using amsmath. 

\href{https://www.ams.org/arc/tex/amsmath/amsldoc.pdf}{https://www.ams.org/arc/tex/amsmath/amsldoc.pdf}

These math expressions would be expected in mathematical physics.

This is not meant to comprehensively represent all expression found in pure math. Also, there are mathematical physics notations not found in pure math -- e.g., Dirac notation.

\section{Algebra}
\begin{equation}
a = b
\end{equation}    

polynomial
\begin{equation}
x^n + y^n = z^n
\end{equation}    

polynomial with subscripts
\begin{equation}
a_1^2 + b_1^2 = c_1^2 
\end{equation}    

\begin{equation}
k_{n+1}=n^{2}+k_{n}^{2}-k_{n-1}
\end{equation}    

\begin{equation}
\cos(2\theta\phi)=\cos^{2}\theta\phi-\sin^{2}\theta\phi
 \end{equation}    

\begin{equation}
\sqrt{|xy|}\leq\left|\frac{x+y}{2}\right|
 \end{equation}    

function of a variable
\begin{equation}
f(x) = y^2
\end{equation}    

With constraint
% https://github.com/allofphysicsgraph/proofofconcept/issues/227
\begin{equation}
K = G + 1 \ {\rm when}\ K << G
\end{equation}    

Equation with range:
\begin{equation}
f(x) = x^2 \ {\rm when}\ x>0 
\end{equation}    

\begin{equation}   
|x|=\begin{cases}
 x  & \text{if }x\geq 0\\\
-x  & \text{if }x< 0
\end{cases}
\end{equation}


\begin{equation}
P_{r-j}=\begin{cases}
0& \text{if $r-j$ is odd},\\
r!\,(-1)^{(r-j)/2}& \text{if $r-j$ is even}.
\end{cases}
\end{equation}

%https://www.authorea.com/users/3/articles/165181-latex-mathematics-examples
continued fractions
\begin{equation} 
x = a_0 + \frac{1}{a_1 + \frac{1}{a_2 + \frac{1}{a_3 + a_4}}}
\end{equation}    

As the fractions continue, they get smaller. If you want to keep the size consistent, use the display style
\begin{equation} 
  x = a_0 + \frac{1}{\displaystyle a_1
          + \frac{1}{\displaystyle a_2
          + \frac{1}{\displaystyle a_3 + a_4}}}
\end{equation}    

system of equations
\begin{equation}    
u=\frac{-y}{x^2+y^2}\,,\quad
v=\frac{x}{x^2+y^2}\,,\quad\text{and}\quad
w=0
\end{equation}


Multi-line equation:
\begin{equation}
\begin{split}
a = x + \\
 c + b * \\
 (y + z)
\end{split}
\end{equation}    

Multi-line equation with alignment:
\begin{equation}
\begin{split}
a = x - \\
& c + b - \\
& f + k
\end{split}
\end{equation}

Simultaneous equations
\begin{equation}
\begin{cases}
3x + 5y + z &= 1 \\
7x - 2y + 4z &= 2 \\
-6x + 3y + 2z &= 3
\end{cases}
\end{equation}

% https://gitmind.com/faq/formula-list.html
fractions, nested
\begin{equation}
\dfrac{ \tfrac{1}{2}[1-(\tfrac{1}{2})^n] }{ 1-\tfrac{1}{2} } = s_n
\end{equation}



binomial
\begin{equation}
\frac{n!}{k!(n-k)!} = \binom{n}{k}
\end{equation}

% https://www.authorea.com/users/3/articles/165181-latex-mathematics-examples
The Taylor series expansion for $e^x$ is 
\begin{equation}
1 + x + \frac{x^2}{2} + \frac{x^3}{6} + \cdots = \sum_{n\geq 0} \frac{x^n}{n!}
\end{equation}

For any non-negative integer $n$,
\begin{equation}
(1+x)^n = \sum_{i=0}^n \binom{n}{i} x^i
\end{equation}


\section{Calculus}
limit
\begin{equation}
\lim_{n \to \infty}x_n = 5
\end{equation}


integration
\begin{equation}
f(x) = \int^a_b \frac{1}{3}x^3
\end{equation}


% https://gitmind.com/faq/formula-list.html
\begin{equation}
\iiint_{E}^{V} \, dx\,dy\,dz = 2
\end{equation}

\begin{equation}
\oint_{C} x^3\, dx + 4y^2\, dy = 6
\end{equation}

partial differential
\begin{equation}
\int_{a}^{b}u\frac{d^{2}v}{dx^{2}}\,dx=\left.u\frac{dv}{dx}\right|_{a}^{b}-\int_{a}^{b}\frac{du}{dx}\frac{dv}{dx}\,dx
 \end{equation}

\section{Linear Algebra}
vector and matrix notation
\begin{equation}
\vec{x} = \hat{A}
\end{equation}    


row vector:
\begin{equation}
\vec{x} = \left[ 0 \ 3 \ y \right]
\end{equation}

column vector:
\begin{align}
    \vec{x} &= \begin{bmatrix}
           x_{1} \\
           x_{2} \\
           x_{3}
         \end{bmatrix}
\end{align}

column vector, arbitrary size:
\begin{align}
    \vec{x} &= \begin{bmatrix}
           x_{1} \\
           x_{2} \\
           \vdots \\
           x_{3}
         \end{bmatrix}
\end{align}


matrix:
\begin{align}
    \hat{x} &= \begin{bmatrix}
           x_{1,1} & x_{1,2} \\
           x_{2,1} & x_{2,2}
         \end{bmatrix}
\end{align}

matrix, arbitrary size:
\begin{align}
    \hat{x} &= \begin{bmatrix}
x_{1,1} & x_{1,2} & \hdots & x_{1,n} \\
x_{2,1} & x_{2,2} \\
\vdots  &         & \ddots &     \\
x_{m,1} &         &        & x_{m,n}
               \end{bmatrix}
\end{align}

% https://tex.stackexchange.com/a/153616/235813
matrix math:
  \begin{align}
    y &= (x_{1},x_{2},\cdots, x_{N})
        \begin{pmatrix}
          \begin{bmatrix}
           ax_{0} + bx_{1} \\           
           \vdots \\
           ax_{n-1}+bx_{n}
          \end{bmatrix} -
          \begin{bmatrix}
           z_{1} \\
           \vdots \\
           z_{n}
         \end{bmatrix}
    \end{pmatrix}
  \end{align}

\section{Quantum: Dirac Notation}

bra
% https://tex.stackexchange.com/a/553176/235813
\begin{equation}
\langle \psi |
\end{equation}


ket
% https://tex.stackexchange.com/a/553176/235813
\begin{equation}
| \phi \rangle
\end{equation}


% https://tex.stackexchange.com/a/553176/235813
\begin{equation}
\langle \psi | \phi \rangle
\end{equation}


\begin{equation}
\langle \psi | A | \phi \rangle
\end{equation}


expectation
\begin{equation}
\langle A \rangle
\end{equation}

\section{Einstein summation notation}

% https://math.stackexchange.com/questions/2429838/einstein-summation-notation-and-kronecker-delta-problem

\href{https://en.wikipedia.org/wiki/Einstein_notation}{Einstein notation}

\begin{equation}
C_{i,j} = A_{i,k}\ B_{k,j}
\end{equation}

The \href{https://en.wikipedia.org/wiki/Kronecker_delta}{Kronecker delta}
\begin{equation}
\sum _{j}\delta _{ij}a_{j}=a_{i}
\end{equation}


\begin{equation}
\delta _{nm}=\lim _{N\to \infty }{\frac {1}{N}}\sum _{k=1}^{N}e^{2\pi i{\frac {k}{N}}(n-m)}
\end{equation}


\end{document}
